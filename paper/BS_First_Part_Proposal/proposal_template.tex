%% ============================================
%% ================ Preambule =================
%% ============================================
\documentclass[]{scrartcl}
\usepackage[margin = 0.5in]{geometry}

\usepackage[pdftex,unicode, 
colorlinks=true,
linkcolor = blue]{hyperref}	% нумерование страниц, ссылки!!!!ИМЕННО В ТАКОМ ПОРЯДКЕ СО СЛЕДУЮЩИМ ПАКЕТОМ
%\usepackage[warn]{mathtext}				% Поддержка русского текста в формулах
\usepackage[T1, T2A]{fontenc}			% Пакет выбора кодировки и шрифтов
\usepackage[utf8]{inputenc} 			% любая желаемая кодировка
\usepackage[english]{babel}		% поддержка русского языка
\usepackage{wrapfig}					% Плавающие картинки
\usepackage{amssymb, amsmath}			% стилевой пакет для формул
\usepackage{algorithm}
\usepackage{algorithmic} 


\ifpdf
\usepackage{cmap} 				% чтобы работал поиск по PDF
\usepackage[pdftex]{graphicx}
%\usepackage{pgfplotstable}		% Для вставки таблиц.
\pdfcompresslevel=9 			% сжимать PDF
\else
\usepackage{graphicx}
\fi

\graphicspath{{./figures/}}
\usepackage{subcaption}
%% ============================================
%% ================ Info =================
%% ============================================
\title{Использование тензорных сетей для сжатия и восстановления изображений}
\author{\begin{tabular}{c c}
	  	 Александр Моложавенко &  molozhavenko.aa@phystech.edu\\
		\end{tabular}}
\date{Project Proposal}

\begin{document}

\maketitle

\begin{abstract}
В этом проекте рассматривается возможность использования тензорных сетей для задач восстановления изображения по значениям в небольшом числе пикселей (tensor completion), а также для сжатия изображений. 
\end{abstract}

\section{Идея}
    
    С помощью тензорных сетей (\textit{PEPS}, \textit{Tensor Ring}, \textit{Tensor Train}),  алгоритма \textit{Greedy-TN} \cite{DBLP:journals/corr/abs-2008-05437}, его модифи-каций и алгоритмами \textit{HOSVD, ALS} найти оптимальные конфигурации для минимально возможного по количеству параметров представления изображений. Аналогично этими методами решается задача восста-новления изображений.
    
\subsection{Проблема}
    
    Пусть имеется картинка $M \in \mathbb{R}^{3 \times d_w \times d_h}$, задаваемая тремя матрицами пикселей (фильтр синего, фильтр красного и фильтр зеленого) или трехмерным тензором.Преобразуем трёхиндексную матрицу $M$ в много индексную матрицу (в тензор высокого порядка) по основанию $b$, то есть:

$$
M \in \mathbb{R}^{3 \times d_w \times d_h} \mapsto \mathcal{T} \in \mathbb{R}^{b \times b \times \dots \times b}.
$$


    Количество мод полученного тензора, равняется $N = \log_b (3\cdot d_h \cdot d_w)$. Количество элементов в таком представлении изображение есть $3 \cdot d_w \cdot d_h = b ^ N$. Однако, если к полученному тензору высокого порядка применить \textit{Tensor Train Decomposition} из $N$ тензоров свертке с максимальном рангом тензорного поезда $m$, то придется хранить всего лишь около $N\cdot b \cdot m^2$ элементов. Что при правильной оптимизации ранга разлажения может дать существенное сжатие.

В терминах статьи  \cite{DBLP:journals/corr/abs-2008-05437} введем в рассмотрение тензорную сеть $\text{TN}(\mathcal{G}^{(1)}, \mathcal{G}^{(2)}, \dots, \mathcal{G}^{(N)})$, которая в свертке по заданным в ней модам даёт тензор $\mathcal{W} \in \mathbb{R}^{b \times b \times \dots \times b}$. В этих обозначениях проблемы восстановления и сжатия изображений перепишутся следующим образом:

\subsubsection{Image compression}
\begin{equation}\label{ImComprProbStatement}
\|\mathcal{T} - \text{TN}(\mathcal{G}^{(1)}, \mathcal{G}^{(2)}, \dots, \mathcal{G}^{(N)})\|_F^2 \to \min\limits_{\mathcal{G}^{(1)}, \mathcal{G}^{(2)}, \dots, \mathcal{G}^{(N)}},
\end{equation}


причём в зависимости от выбранного метода оптимизации данной функции будут или не будут наклады-ваться дополнительные условия связи между \textit{core}-тензорами тензорной сети (адаптивный метод \textit{Greedy-TN}, например автоматически подбирает нужные связи)

\subsubsection{Image Completion}
\begin{equation}\label{ImComplProbStatement}
\frac{1}{|\Omega|}\sum\limits_{(i_1, \dots, i_N) \in \Omega}\left(\mathcal{T}_{i_1, \dots, i_N} - \text{TN}(\mathcal{G}^{(1)}, \mathcal{G}^{(2)}, \dots, \mathcal{G}^{(N)})_{i_1, \dots, i_N}\right)^2 \to \min\limits_{\mathcal{G}^{(1)}, \mathcal{G}^{(2)}, \dots, \mathcal{G}^{(N)}},
\end{equation}

где $\Omega$ -- множество индексов, значения под которыми нам известны (индексы известных частей картинки после её отображения в тензор высокого порядка).

\newpage
\section{Литературный обзор}
\subsection{Tensor Ring}


В работе \cite{DBLP:journals/corr/ZhaoZXZC16} исследуются свойства \textit{Tensor Ring Decomposition}, которое заключается в представлении тензора высокого порядка в виде последовательности трёхмерных тензоров, которые умножаются <<по кругу>>. Формально, пусть $\mathcal{T} \in \mathbb{R}^{n_1 \times n_2 \times \dotsm \times n_d}$ -- тензор высокого порядка, а $\mathcal{Z}_k \in \mathbb{R}^{r_k \times n_k \times r_{k + 1}}, k \in \overline{1, \dotsc, d}$ -- множество трёхмерных тензоров, такие что:

\begin{equation}\label{TRDecomp}
    \mathcal{T}(i_1, i_2, \dotsc, i_d) = \text{Tr}\{(\mathcal{Z}_1(i_1)\mathcal{Z}_2(i_2) \dotsm Z_d(i_d))\} = \text{Tr}\left\{\prod_{k=1}^d \mathcal{Z}_k(i_k)\right\},
\end{equation}

где $\mathcal{T}(i_1, i_2, \dotsc, i_d)$ -- число в многоиндексной таблице $\mathcal{T}$ с  индексами $i_1, i_2, \dotsc, i_d$, а  $\mathcal{Z}_k(i_k), \forall k \in \overline{1, \dotsc, d}$ -- срез трехмерного тензора $\mathcal{Z}_k$ с размерностью  $r_k \times r_{k + 1}$:

\begin{figure}[h!tp]
    \centering
    \includegraphics[scale=0.5]{TensorRing/DecompTR.PNG}
    \caption{Графическое представление \textit{Tensor Ring Decomposition}}
    \label{fig:TRDecomp}
\end{figure}


\textit{Tensor Ring Decomposition} в отличие от \textit{Tensor Train Decomposition} <<устойчиво>> по отношению к циклическим перестановкам рёбер тензора:


\textbf{Теорема}


Пусть $\mathcal{T} \in \mathbb{R}^{n_1 \times n_2 \times \dotsm \times n_d}$ -- тензор высокого порядка с \textit{Tensor Ring Decomposition} $\mathcal{R}(\mathcal{Z}_1, \mathcal{Z}_2, \dotsc, \mathcal{Z}_d)$. Определим тензор $\vec{\mathcal{T}} \in \mathbb{R}^{n_{k + 1} \times \dotsm \times n_d \times n_1 \times \dotsm \times n_k}$, полученный циклическим сдвигом рёбер исходного тензора $\mathcal{T}$ на $k$, тогда его \textit{Tensor Ring Decomposition} будет $\mathcal{R}(\mathcal{Z}_{k + 1}, \dotsc, \mathcal{Z}_d,  \mathcal{Z}_1, \dotsc, \mathcal{Z}_k)$

Также статья даёт ответы на вопросы о некоторых алгебраических свойствах разложения \textit{Tensor Ring} таких, как сложение и некоторые <<хитрые>> умножения.
\newpage
\subsection{Tensor Ring}


В работе \cite{DBLP:journals/corr/ZhaoZXZC16} исследуются свойства \textit{Tensor Ring Decomposition}, которое заключается в представлении тензора высокого порядка в виде последовательности трёхмерных тензоров, которые умножаются <<по кругу>>. Формально, пусть $\mathcal{T} \in \mathbb{R}^{n_1 \times n_2 \times \dotsm \times n_d}$ -- тензор высокого порядка, а $\mathcal{Z}_k \in \mathbb{R}^{r_k \times n_k \times r_{k + 1}}, k \in \overline{1, \dotsc, d}$ -- множество трёхмерных тензоров, такие что:

\begin{equation}\label{TRDecomp}
    \mathcal{T}(i_1, i_2, \dotsc, i_d) = \text{Tr}\{(\mathcal{Z}_1(i_1)\mathcal{Z}_2(i_2) \dotsm Z_d(i_d))\} = \text{Tr}\left\{\prod_{k=1}^d \mathcal{Z}_k(i_k)\right\},
\end{equation}

где $\mathcal{T}(i_1, i_2, \dotsc, i_d)$ -- число в многоиндексной таблице $\mathcal{T}$ с  индексами $i_1, i_2, \dotsc, i_d$, а  $\mathcal{Z}_k(i_k), \forall k \in \overline{1, \dotsc, d}$ -- срез трехмерного тензора $\mathcal{Z}_k$ с размерностью  $r_k \times r_{k + 1}$:

\begin{figure}[h!tp]
    \centering
    \includegraphics[scale=0.5]{TensorRing/DecompTR.PNG}
    \caption{Графическое представление \textit{Tensor Ring Decomposition}}
    \label{fig:TRDecomp}
\end{figure}


\textit{Tensor Ring Decomposition} в отличие от \textit{Tensor Train Decomposition} <<устойчиво>> по отношению к циклическим перестановкам рёбер тензора:


\textbf{Теорема}


Пусть $\mathcal{T} \in \mathbb{R}^{n_1 \times n_2 \times \dotsm \times n_d}$ -- тензор высокого порядка с \textit{Tensor Ring Decomposition} $\mathcal{R}(\mathcal{Z}_1, \mathcal{Z}_2, \dotsc, \mathcal{Z}_d)$. Определим тензор $\vec{\mathcal{T}} \in \mathbb{R}^{n_{k + 1} \times \dotsm \times n_d \times n_1 \times \dotsm \times n_k}$, полученный циклическим сдвигом рёбер исходного тензора $\mathcal{T}$ на $k$, тогда его \textit{Tensor Ring Decomposition} будет $\mathcal{R}(\mathcal{Z}_{k + 1}, \dotsc, \mathcal{Z}_d,  \mathcal{Z}_1, \dotsc, \mathcal{Z}_k)$

Также статья даёт ответы на вопросы о некоторых алгебраических свойствах разложения \textit{Tensor Ring} таких, как сложение и некоторые <<хитрые>> умножения.
\subsection{Tensor Ring}


В работе \cite{DBLP:journals/corr/ZhaoZXZC16} исследуются свойства \textit{Tensor Ring Decomposition}, которое заключается в представлении тензора высокого порядка в виде последовательности трёхмерных тензоров, которые умножаются <<по кругу>>. Формально, пусть $\mathcal{T} \in \mathbb{R}^{n_1 \times n_2 \times \dotsm \times n_d}$ -- тензор высокого порядка, а $\mathcal{Z}_k \in \mathbb{R}^{r_k \times n_k \times r_{k + 1}}, k \in \overline{1, \dotsc, d}$ -- множество трёхмерных тензоров, такие что:

\begin{equation}\label{TRDecomp}
    \mathcal{T}(i_1, i_2, \dotsc, i_d) = \text{Tr}\{(\mathcal{Z}_1(i_1)\mathcal{Z}_2(i_2) \dotsm Z_d(i_d))\} = \text{Tr}\left\{\prod_{k=1}^d \mathcal{Z}_k(i_k)\right\},
\end{equation}

где $\mathcal{T}(i_1, i_2, \dotsc, i_d)$ -- число в многоиндексной таблице $\mathcal{T}$ с  индексами $i_1, i_2, \dotsc, i_d$, а  $\mathcal{Z}_k(i_k), \forall k \in \overline{1, \dotsc, d}$ -- срез трехмерного тензора $\mathcal{Z}_k$ с размерностью  $r_k \times r_{k + 1}$:

\begin{figure}[h!tp]
    \centering
    \includegraphics[scale=0.5]{TensorRing/DecompTR.PNG}
    \caption{Графическое представление \textit{Tensor Ring Decomposition}}
    \label{fig:TRDecomp}
\end{figure}


\textit{Tensor Ring Decomposition} в отличие от \textit{Tensor Train Decomposition} <<устойчиво>> по отношению к циклическим перестановкам рёбер тензора:


\textbf{Теорема}


Пусть $\mathcal{T} \in \mathbb{R}^{n_1 \times n_2 \times \dotsm \times n_d}$ -- тензор высокого порядка с \textit{Tensor Ring Decomposition} $\mathcal{R}(\mathcal{Z}_1, \mathcal{Z}_2, \dotsc, \mathcal{Z}_d)$. Определим тензор $\vec{\mathcal{T}} \in \mathbb{R}^{n_{k + 1} \times \dotsm \times n_d \times n_1 \times \dotsm \times n_k}$, полученный циклическим сдвигом рёбер исходного тензора $\mathcal{T}$ на $k$, тогда его \textit{Tensor Ring Decomposition} будет $\mathcal{R}(\mathcal{Z}_{k + 1}, \dotsc, \mathcal{Z}_d,  \mathcal{Z}_1, \dotsc, \mathcal{Z}_k)$

Также статья даёт ответы на вопросы о некоторых алгебраических свойствах разложения \textit{Tensor Ring} таких, как сложение и некоторые <<хитрые>> умножения.
\subsection{Tensor Ring}


В работе \cite{DBLP:journals/corr/ZhaoZXZC16} исследуются свойства \textit{Tensor Ring Decomposition}, которое заключается в представлении тензора высокого порядка в виде последовательности трёхмерных тензоров, которые умножаются <<по кругу>>. Формально, пусть $\mathcal{T} \in \mathbb{R}^{n_1 \times n_2 \times \dotsm \times n_d}$ -- тензор высокого порядка, а $\mathcal{Z}_k \in \mathbb{R}^{r_k \times n_k \times r_{k + 1}}, k \in \overline{1, \dotsc, d}$ -- множество трёхмерных тензоров, такие что:

\begin{equation}\label{TRDecomp}
    \mathcal{T}(i_1, i_2, \dotsc, i_d) = \text{Tr}\{(\mathcal{Z}_1(i_1)\mathcal{Z}_2(i_2) \dotsm Z_d(i_d))\} = \text{Tr}\left\{\prod_{k=1}^d \mathcal{Z}_k(i_k)\right\},
\end{equation}

где $\mathcal{T}(i_1, i_2, \dotsc, i_d)$ -- число в многоиндексной таблице $\mathcal{T}$ с  индексами $i_1, i_2, \dotsc, i_d$, а  $\mathcal{Z}_k(i_k), \forall k \in \overline{1, \dotsc, d}$ -- срез трехмерного тензора $\mathcal{Z}_k$ с размерностью  $r_k \times r_{k + 1}$:

\begin{figure}[h!tp]
    \centering
    \includegraphics[scale=0.5]{TensorRing/DecompTR.PNG}
    \caption{Графическое представление \textit{Tensor Ring Decomposition}}
    \label{fig:TRDecomp}
\end{figure}


\textit{Tensor Ring Decomposition} в отличие от \textit{Tensor Train Decomposition} <<устойчиво>> по отношению к циклическим перестановкам рёбер тензора:


\textbf{Теорема}


Пусть $\mathcal{T} \in \mathbb{R}^{n_1 \times n_2 \times \dotsm \times n_d}$ -- тензор высокого порядка с \textit{Tensor Ring Decomposition} $\mathcal{R}(\mathcal{Z}_1, \mathcal{Z}_2, \dotsc, \mathcal{Z}_d)$. Определим тензор $\vec{\mathcal{T}} \in \mathbb{R}^{n_{k + 1} \times \dotsm \times n_d \times n_1 \times \dotsm \times n_k}$, полученный циклическим сдвигом рёбер исходного тензора $\mathcal{T}$ на $k$, тогда его \textit{Tensor Ring Decomposition} будет $\mathcal{R}(\mathcal{Z}_{k + 1}, \dotsc, \mathcal{Z}_d,  \mathcal{Z}_1, \dotsc, \mathcal{Z}_k)$

Также статья даёт ответы на вопросы о некоторых алгебраических свойствах разложения \textit{Tensor Ring} таких, как сложение и некоторые <<хитрые>> умножения.

\section{Метрики качества}
\begin{enumerate}
    \item[1)] Относительное сжатие картинок для задачи сжатия
    \item[2)] Относительная ошибка восстановления картинок для задачи восстановления 
    \item[3)] PSNR
    \item[4)] Квадрат нормы Фробениуса разности тензоров
\end{enumerate}

\section{Примерный план}
\begin{itemize}
	\item Для задачи сжатия и восстановления: построить простую тензорную сеть для тензоризированной двумя различными методами картинки, используя  \textit{TT-SVD}, \textit{ALS} и \textit{Adam}.
	\item Проанализировать полученные результаты с помощью объявленных метрик.
	\item Сравнить работу алгоритма сжатия и восстановления с методами, работающими без тензоризации, например: \textit{SVD} и скелетное разложение. 
\end{itemize}
\bibliographystyle{unsrt}
\bibliography{biblio}

\end{document}
